\subsection{Parameterized algorithms}

\begin{frame}{Distance to (co-)cluster}
    \begin{block}{}
        Why investigate these parameters? \onslide<2->{Because $\tau(G) \geq \max\{\dc(G), \dcc(G)\}$.}
    \end{block}
    \onslide<3->
    \fullcite{matching_cut_ipec}
\end{frame}

\begin{frame}{Distance to (co-)cluster}
    \begin{block}{}
        $G$ is a cluster graph if each of its connected components is a clique (cluster).
    \end{block}
    \onslide<2->
    \begin{figure}[!htb]
        \centering
        \hspace{-0.5cm}
        \begin{tikzpicture}[scale=0.6]
            \GraphInit[unit=3,vstyle=Normal]
            \SetVertexNormal[Shape=circle, FillColor=black, MinSize=1pt]
            \tikzset{VertexStyle/.append style = {inner sep = 1.2, outer sep = \outers}}
            \SetVertexNoLabel
            \begin{scope}[shift={(-3, 2)}]
                \grComplete[RA=1,prefix=t]{3}
            \end{scope}
            
            \begin{scope}[shift={(3, 2)}]
                \grComplete[RA=1,prefix=s]{4}
            \end{scope}
            
            \begin{scope}[shift={(-1, 3)}]
                \grComplete[RA=1,prefix=v]{1}
            \end{scope}
            
            \onslide<2-3>{
                \begin{scope}[shift={(-1.5, 0)}]
                    \grPath[RA=1, prefix=p]{4}
                    \Edge[style=bend right](p0)(p2)
                    \Edge[style=bend right](p1)(p3)
                \end{scope}
                \Edge(v0)(p1)
                \Edge(v0)(p2)
                
                \Edge(s2)(p2)
                \Edge(s3)(p3)
                
                \Edge(t0)(p0)
                \Edge(t2)(p0)
                
                
                \onslide<3> {
                    \draw (0,0) ellipse (2cm and 1cm);
                    \node[color=red] at (0, -1.5) {$U$};
                }
            }
        \end{tikzpicture}
    \end{figure}
    \onslide<3->
    \begin{block}{}
        The modulator $U \subset V(G)$ is a set such that $G - U$ is a cluster graph, with clusters $\{C_1, \dots, C_{\ell}\}$.
    \end{block}
    \onslide<5>
    \begin{block}{}
        A co-cluster graph is the complement graph of a cluster graph.
    \end{block}
\end{frame}


\begin{frame}{The basics}
    \begin{block}{Key idea}
        Partition $U$ in monochromatic sets $\{U_1, \dots, U_\ell\}$ and merge them until we get a kernel.
    \end{block}
\end{frame}

\begin{frame}{$N^{2d}(U_i)$}
    \vspace{-0.3cm}
    \begin{block}{}
        For each $U_i$, a vertex $v \in V(G) \setminus U$ is in $N^{2d}(U_i)$ if:
        \begin{itemize}
            \item<2->[1] $v$ has at least $d+1$ neighbors in $U_i$; or
            \item<3->[2] $v$ is in a cluster $C$ of size at least $2d+1$ in $G - U$ such that there is some vertex of $C$ with at least $d+1$ neighbors in $U_i$; or
            \item<4->[3] $v$ is in a cluster $C$ and some vertex in $U_i$ has $2d$ neighbors in $C$; or
            \item<5->[4] $v$ is in a cluster $C$ of size at least $2d+1$ and there is some vertex in $U_i$ with $d+1$ neighbors in $C$.
        \end{itemize}
    \end{block}
    \begin{figure}[!htb]
        \centering
        \hspace*{1cm}\begin{tikzpicture}[scale=0.6]
            \GraphInit[unit=3,vstyle=Normal]
            \SetVertexNormal[Shape=circle, FillColor=black, MinSize=1pt]
            \tikzset{VertexStyle/.append style = {inner sep = 1.2, outer sep = \outers}}
            \SetVertexNoLabel
            \node at (-2.6, 0) {$U_i$};
            \node at (3, 0) {$d=2$};
            \draw (0,0) ellipse (2cm and 0.3cm);
            \Vertex[x=-1.2, y=0]{x0}
            \Vertex[x=-0.4, y=0]{x1}
            \Vertex[x=0.4, y=0]{x2}
            \Vertex[x=1.2, y=0]{x3}
            
            \onslide<2->{
                \Vertex[x=-3.5, y=-2.3]{c1}
                \Edge(c1)(x0)
                \Edge(c1)(x1)
                \Edge(c1)(x2)
            }
            
            \onslide<3-> {
                \begin{scope}[scale=0.8, shift={(-1.5,-4)}]
                    \tikzset{VertexStyle/.append style = {inner sep = 1.2pt, outer sep = \outers}}
                    \begin{scope}[rotate=90]
                        \grComplete[RA=1, prefix=p]{5}
                    \end{scope}
                    \Edge(p0)(x0)
                    \Edge(p0)(x1)
                    \Edge(p0)(x2)
                \end{scope}
            }
            
            \onslide<4->{
                \begin{scope}[scale=0.8, shift={(2,-4)}]
                    \tikzset{VertexStyle/.append style = {inner sep = 1.2pt, outer sep = \outers}}
                    \begin{scope}[rotate=75]
                        \grComplete[RA=1, prefix=t]{6}
                    \end{scope}
                    \Edge(x2)(t0)
                    \Edge(x2)(t1)
                    \Edge[style = bend left](x2)(t5)
                    \Edge(x2)(t2)
                \end{scope}
                
                \onslide<5->
                \begin{scope}[scale=0.8, shift={(6,-3)}]
                    \tikzset{VertexStyle/.append style = {inner sep = 1.2pt, outer sep = \outers}}
                    \begin{scope}[rotate=75]
                        \grComplete[RA=1, prefix=q]{5}
                    \end{scope}
                    \Edge(x3)(q0)
                    \Edge(x3)(q1)
                    \Edge(x3)(q2)
                \end{scope}
            }
        \end{tikzpicture}
    \end{figure}
\end{frame}

\begin{frame}{The kernel}
    
    \begin{theorem}
        When parameterized by distance to cluster, \pname{$d$-Cut} admits a polynomial kernel with $\bigO{d^2\dc(G)^{2d+1}}$ vertices that can be computed in $\bigO{d^4\dc(G)^{2d+1}(n+m)}$ time.
    \end{theorem}

\end{frame}

\begin{frame}{Number of crossing edges}
    \begin{block}{}
        We can solve \pname{$d$-Cut} in $\FPT$ time using the treewidth reduction technique.
    \end{block}
    \fullcite{marx_treewidth_reduction}
    \pause
    \begin{block}{}
        We cast \pname{$d$-Cut} as a separation problem on the line graph: for some pair $s,t \in V(L(G))$, we want a cutset with a maximum clique of size $\leq d$.
    \end{block}
    \pause
    \begin{theorem}{}
      \textsc{$d$-Cut} is \FPT\ paramterized by the number of crossing edges.
    \end{theorem}
\end{frame}

\begin{frame}{Treewidth}
    \begin{block}{}
        Dynamic programming on a nice tree decomposition.
    \end{block}
    \begin{columns}[T]
        \begin{column}{0.4\textwidth}
            \vspace{-0.5cm}
            \begin{figure}[!htb]
                \centering
                \begin{tikzpicture}[rotate = 0]
                        %\draw[help lines] (-5,-5) grid (5,5);
                        \GraphInit[unit=3,vstyle=Normal]
                        \SetVertexNormal[Shape=circle, MinSize=3pt]
                        \tikzset{VertexStyle/.append style = {inner sep = \inners, outer sep = \outers}}
                        \SetVertexNoLabel
                        \begin{scope}[rotate=90]
                            \draw[fill=gray!20] (0,0) circle (1.2);
                            \node at (1.5, 0) {$V_x$};
                            \grComplete[RA=0.816747, prefix=t]{3}
                            \SetVertexLabel
                            \Vertex[Node, L = {0}, Math]{t0}
                        \end{scope}
                        \begin{scope}[rotate=90]
                            \tikzset{VertexStyle/.append style = {inner sep = 3pt, shape = rectangle}}
                            \SetVertexLabel
                            \Vertex[Node, L = {2}, Math]{t2}
                            \Vertex[Node, L = {1}, Math]{t1}
                        \end{scope}
                        \begin{scope}[rotate=45, shift={(-1.71424, -1.71424)}]
                            \grCycle[RA=1, prefix=c]{4}
                            \Edge(t0)(t2)
                        \end{scope}
                        \begin{scope}[rotate=45, shift={(-1.71424, -1.71424)}]
                            \SetVertexNormal[Shape=circle, FillColor = black, MinSize=3pt]
                            \tikzset{VertexStyle/.append style = {shape = rectangle, inner sep = 3pt, outer sep = \outers}}
                            \Vertex[Node]{c3}
                        \end{scope}
                        \begin{scope}[rotate=45, shift={(-1.71424, -1.71424)}]
                            \SetVertexNormal[Shape=circle, FillColor = black, MinSize=3pt]
                            \tikzset{VertexStyle/.append style = {inner sep = 2pt, outer sep = \outers}}
                            \Vertex[Node]{c0}
                            \Vertex[Node]{c1}
                            \Vertex[Node]{c2}
                        \end{scope}
                        \Edge(c0)(t2)
                        \Edge(c1)(t2)
                        \Edge(c1)(t1)
                        \Edge(c0)(c2)
                        %\AssignVertexLabel{t}{0,1,0}
                \end{tikzpicture}
            \end{figure}
        \end{column}
        \begin{column}{0.55\textwidth}
            \begin{itemize}
                \item<2-> $S \subseteq V_x$ represents which (squared) elements of $V_x$ are assigned to $A$.
                \item<3-> $\mathbf{\alpha} \in \{0, \dots, d\}^{|V_x|}$ stores how many neighbors each $v \in V_x$ has across the cut \textbf{\textit{outside}} of $V_x$.
                \item<4-> $t = 1$ if $A$ and $B$ are non-empty. 
                \item<5-> $f_x(S, \mathbf{\alpha}, t) = 1$ iff the subtree rooted at bag $x$ has a partition that satisfies all of the above.
            \end{itemize}
        \end{column}
    \end{columns}
\end{frame}

\begin{frame}{Treewidth}
    \begin{theorem}
        For every $d \geq 1$, \pname{$d$-Cut} can be solved in $\bigOs{2^{\tw(G)}(d+1)^{2\tw(G)}}$.
    \end{theorem}
    \pause
    \begin{corollary}
        \pname{Matching Cut} can be solved in $\bigOs{8^{\tw(G)}}$. (Improvement upon $12^{\tw(G)}$).
    \end{corollary}
    \fullcite{matching_cut_structural}
\end{frame}

\begin{frame}{Other results}
    \begin{theorem}
        \pname{$d$-Cut} has no polynomial kernel parameterized by treewidth, maximum degree and number of crossing edges, unless $\NP \subseteq \coNP / \poly$.
    \end{theorem}
    \pause
    \begin{theorem}
        \pname{$d$-Cut} can be solved in $\bigOs{(d+1)^{\dc(G)}}$ time.
    \end{theorem}
    \pause
    \begin{theorem}
        \pname{$d$-Cut} can be solved in $\bigOs{(d+1)^{\dcc(G)}}$ time.
    \end{theorem}
\end{frame}

\begin{frame}{Open Problems}
    \begin{block}{}
        Settle the complexity for graphs of maximum degree between $d+3$ and $2d+1$.
        We guess most of these should be solvable in polynomial time, but right now we have no idea how to tackle this problem.
    \end{block}
    \pause
    \begin{block}{}
        Decide if the exponential dependency on $d$ is necessary, or if we can restrict its influence to a polynomial factor.
    \end{block}
    \pause
    \begin{block}{}
        Explore the other generalizations of \pname{Matching Cut}, in particular Option 3, which seems considerably harder than the other two.
    \end{block}
    \begin{block}{Option 3}
        Look for a \textbf{$\boldsymbol{p}$-partition}  such that each pair of parts forms a matching cut.
    \end{block}
\end{frame}
