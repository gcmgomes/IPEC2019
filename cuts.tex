\begin{frame}{\pname{Matching Cut}}
    \begin{block}{}
        A matching cut is a bipartition of $V(G)$ such that each vertex has at most one neighbor in the other part.
    \end{block}
    \pause
    
    \begin{figure}[!htb]
        \centering
        \hspace{-0.5cm}
        \begin{tikzpicture}[scale=1]
                %\draw[help lines] (-5,-5) grid (5,5);
                \GraphInit[unit=3,vstyle=Simple]
                \SetVertexSimple[Shape=circle, FillColor=black, MinSize=2pt]
                \tikzset{VertexStyle/.append style = {inner sep = \inners, outer sep = \outers}}
                \SetVertexNoLabel
                \grCycle[RA=1.5, prefix=c]{4}
                \AddVertexColor{white}{c0,c3}
                \AddVertexColor{black!50}{c1,c2}
        \end{tikzpicture}
        \pause
        \hspace{1.5cm}
        \begin{tikzpicture}[scale=1,rotate=90]
                %\draw[help lines] (-5,-5) grid (5,5);
                \GraphInit[unit=3,vstyle=Simple]
                \SetVertexSimple[Shape=circle, FillColor=black, MinSize=2pt]
                \tikzset{VertexStyle/.append style = {inner sep = \inners, outer sep = \outers}}
                \SetVertexNoLabel
                \grCycle[RA=1.5, prefix=t]{3}
        \end{tikzpicture}
        \pause
        \hspace{1.75cm}
        \begin{tikzpicture}[scale=1, rotate=-90, shift={(0,0)}]
                %\draw[help lines] (-5,-5) grid (5,5);
                \GraphInit[unit=3,vstyle=Simple]
                \SetVertexSimple[Shape=circle, FillColor=black, MinSize=2pt]
                \tikzset{VertexStyle/.append style = {inner sep = \inners, outer sep = \outers}}
                \SetVertexNoLabel
                \grCompleteBipartite[RA=1.5, RB = 1.5, RS=1]{2}{3}
        \end{tikzpicture}
    \end{figure}
    \pause
    \fullcite{matching_cut_graham}
    \begin{block}{}
        Which graphs admit a matching cut?
    \end{block}
\end{frame}

\begin{frame}{\pname{Matching Cut} and graph classes}
    \fullcite{chvatal_matching_cut}
    \pause
    \begin{block}{}
        Recognizing these graphs is $\NPH$ even if $\Delta(G) = 4$...
    \end{block}
    \pause
    \fullcite{matching_cut_planar}
    \begin{block}{}
        ...$G$ is planar...
    \end{block}
    \pause
    \fullcite{stable_cutset_line_graphs}
    \begin{block}{}
        ...or $G$ is bipartite.
    \end{block}
\end{frame}

\begin{frame}{Algorithmic results for \pname{Matching Cut}}
    \fullcite{marx_treewidth_reduction}
    \begin{block}{}
        $\FPT$ parameterized by the number of edges crossing the cut.
    \end{block}
    \pause
    \fullcite{matching_cut_tcs}
    \begin{block}{}
        $\FPT$ parameterized by vertex cover; $\bigOs{2^{n/2}}$ exact exponential algorithm.
    \end{block}
\end{frame}

\begin{frame}{Algorithmic results for \pname{Matching Cut}}
    \fullcite{matching_cut_structural}
    \begin{block}{}
        $\FPT$ parameterized by treewidth, neighborhood diversity, or twin cover.
    \end{block}
\end{frame}


\begin{frame}{Algorithmic results for \pname{Matching Cut}}
    \fullcite{matching_cut_ipec}
    \begin{block}{}
        \begin{itemize}
            \item $\FPT$ parameterized by distance to cluster, distance to co-cluster;
            \item Quadratic kernel for distance to cluster, linear kernel for distance to clique;
            \item No polynomial kernel for treewidth $+$ number of crossing edges $+$ maximum degree (unless $\NP \subseteq \coNP/\poly$).
            \item Exact exponential running in $\bigOs{1.38^n}$.
        \end{itemize}
    \end{block}
    \pause
    \begin{block}{}
        During their presentation, asked for results on the generalization of \pname{Matching Cut} we discuss here.
    \end{block}
\end{frame}

\begin{frame}{Cuts and degree constraints}
    %\setbeamercovered{transparent}
    \onslide<1-4>
    \begin{block}{}
        A matching cut is a bipartition of $V(G)$ such that each vertex has at most one neighbor in the other part.
    \end{block}
    \onslide<2-5>
    \begin{block}{Option 1}
        Look for a \textbf{bipartition} such that each vertex has at most $\boldsymbol{d}$ neighbors in the other part.
    \end{block}
    \onslide<3-4>
    \begin{block}{Option 2}
        Look for a \textbf{$\boldsymbol{p}$-partition} such that each vertex has at most \textbf{one} neighbor outside its part.
    \end{block}
    \onslide<4-4>
    \begin{block}{Option 3}
        Look for a \textbf{$\boldsymbol{p}$-partition}  such that each pair of parts forms a matching cut.
    \end{block}
    %\setbeamercovered{invisible}
\end{frame}
